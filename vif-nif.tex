% arara: xelatex
\documentclass[12pt]{article}

% \usepackage{physics}

\usepackage{hyperref}
\hypersetup{
    colorlinks=true,
    linkcolor=blue,
    filecolor=magenta,      
    urlcolor=cyan,
    pdftitle={Overleaf Example},
    pdfpagemode=FullScreen,
    }

\usepackage{tikzducks}

\usepackage{tikz} % картинки в tikz
\usetikzlibrary{shapes, arrows, positioning}
\usepackage{microtype} % свешивание пунктуации

\usepackage{array} % для столбцов фиксированной ширины

\usepackage{indentfirst} % отступ в первом параграфе

\usepackage{sectsty} % для центрирования названий частей
\allsectionsfont{\centering}

\usepackage{amsmath, amsfonts, amssymb} % куча стандартных математических плюшек

\usepackage{comment}

\usepackage[top=2cm, left=1.2cm, right=1.2cm, bottom=2cm]{geometry} % размер текста на странице

\usepackage{lastpage} % чтобы узнать номер последней страницы

\usepackage{enumitem} % дополнительные плюшки для списков
%  например \begin{enumerate}[resume] позволяет продолжить нумерацию в новом списке
\usepackage{caption}

\usepackage{url} % to use \url{link to web}


\newcommand{\smallduck}{\begin{tikzpicture}[scale=0.3]
    \duck[
        cape=black,
        hat=black,
        mask=black
    ]
    \end{tikzpicture}}

\usepackage{fancyhdr} % весёлые колонтитулы
\pagestyle{fancy}
\lhead{}
\chead{Виф-ниф, ваф-наф, вуф-нуф и матрица-мать всех регрессий}
\rhead{}
\lfoot{}
\cfoot{}
\rfoot{}

\renewcommand{\headrulewidth}{0.4pt}
\renewcommand{\footrulewidth}{0.4pt}

\usepackage{tcolorbox} % рамочки!

\usepackage{todonotes} % для вставки в документ заметок о том, что осталось сделать
% \todo{Здесь надо коэффициенты исправить}
% \missingfigure{Здесь будет Последний день Помпеи}
% \listoftodos - печатает все поставленные \todo'шки


% более красивые таблицы
\usepackage{booktabs}
% заповеди из докупентации:
% 1. Не используйте вертикальные линни
% 2. Не используйте двойные линии
% 3. Единицы измерения - в шапку таблицы
% 4. Не сокращайте .1 вместо 0.1
% 5. Повторяющееся значение повторяйте, а не говорите "то же"


\setcounter{MaxMatrixCols}{20}
% by crazy default pmatrix supports only 10 cols :)


\usepackage{fontspec}
\usepackage{libertine}
\usepackage{polyglossia}

\setmainlanguage{russian}
\setotherlanguages{english}

% download "Linux Libertine" fonts:
% http://www.linuxlibertine.org/index.php?id=91&L=1
% \setmainfont{Linux Libertine O} % or Helvetica, Arial, Cambria
% why do we need \newfontfamily:
% http://tex.stackexchange.com/questions/91507/
% \newfontfamily{\cyrillicfonttt}{Linux Libertine O}

\AddEnumerateCounter{\asbuk}{\russian@alph}{щ} % для списков с русскими буквами
% \setlist[enumerate, 2]{label=\asbuk*),ref=\asbuk*}

%% эконометрические сокращения
\DeclareMathOperator{\Cov}{\mathbb{C}ov}
\DeclareMathOperator{\Corr}{\mathbb{C}orr}
\DeclareMathOperator{\Var}{\mathbb{V}ar}
\DeclareMathOperator{\col}{col}
\DeclareMathOperator{\row}{row}

\let\P\relax
\DeclareMathOperator{\P}{\mathbb{P}}

\DeclareMathOperator{\E}{\mathbb{E}}
% \DeclareMathOperator{\tr}{trace}
\DeclareMathOperator{\card}{card}

\DeclareMathOperator{\Convex}{Convex}
\DeclareMathOperator{\plim}{plim}

\usepackage{mathtools}
\DeclarePairedDelimiter{\norm}{\lVert}{\rVert}
\DeclarePairedDelimiter{\abs}{\lvert}{\rvert}
\DeclarePairedDelimiter{\scalp}{\langle}{\rangle}
\DeclarePairedDelimiter{\ceil}{\lceil}{\rceil}

\newcommand{\cN}{\mathcal{N}}
\newcommand{\cF}{\mathcal{F}}

\newcommand{\RR}{\mathbb{R}}
\newcommand{\NN}{\mathbb{N}}
\newcommand{\hb}{\hat{\beta}}
\newcommand{\dPois}{\mathrm{Pois}}





\begin{document}
Цель этой заметки — объяснить, что находится в обратной матрице кросс-произведений $(X^TX)^{-1}$.

Поехали! 

Обозначим нашу матрицу $(X^TX)^{-1}$ как $V$. Она симметрична, $V^T = V$. 
По определению, $X^TX \cdot V = I$.
Разобьём произведение на две части по-другому, $X^T \cdot XV = I$.

Заметим, что $X_* = XV$ — это матрица такого же размера, как $X$, $n\times k$. 
Что же собой представляют эти звезданутые иксы, $X_*$?

Во-первых, $X_* = XV$, поэтому новые $X_*$ — это линейные комбинации исходных $X$ с весами из матрицы $V = (X^TX)^{-1}$.
Например, $x_{1*} = v_{11} x_1 + v_{21} x_2 + \dots + v_{k1} x_k$.

Во-вторых, $X_*^T X_* = (X V)^T   X V = V^T X^TX V = V = (X^TX)^{-1}$. 
Матрица кросс-произведений новых переменных является обратной к матрице кросс-произведений исходных переменных. 

В частности, $v_{11} = x_{1*}^T x_{1*}$, а $v_{12} = v_{21} = x_{1*}^T x_{2*}$.

Во-третьих, $XX_* = I$, то есть новая переменная $x_{i*}$ перпендикулярна всем старым переменным кроме $x_i$.
Ортогональность позволяет нам взглянуть на первую звезданутую переменную как на остаток в регресии.
Первая ничем не выделяется, просто индексов меньше будет.
\[
x_{1*} = v_{11} x_1 + v_{21} x_2 + \dots + v_{k1} x_k 
\]
Изолируем $x_1$ слева, а всё остальное — перенесём вправо:
\[
x_1 = -\frac{v_{21}}{v_{11}} x_2  -\frac{v_{31}}{v_{11}} x_3 - \dots -\frac{v_{k1}}{v_{11}} x_k + \frac{1}{v_{11}} x_{1*} 
\]
Мы только что заметили, что переменная $x_{1*}$ перпендикулярна $x_2$, $x_3$, \dots, $x_k$, поэтому перед нами регрессия $x_1$ на все остальные переменные $X_{-1}$:
\[
x_1 = \hb_{12} x_2 + \hb_{13} x_3 + \dots + \hb_{1k} x_k + e_1
\]
Здесь $e_1 = x_{1*} /v_{11}$ — это остаток от регрессии $x_1$ на $x_2$, \dots, $x_k$.
И $\hb_{12} = -{v_{21}}/{v_{11}} = -{v_{12}}/{v_{11}}$ — это оценка коэффиента в регрессии $x_1$ перед переменной $x_2$.

Отсюда мы сразу видим, что $V = (X^TX)^{-1}$ — это \textbf{матрица-мать всех регрессий!}
Хочешь узнать коэффициент в регрессии $x_1$ на все остальные переменные перед $x_7$?
В уме, $\hb_{17} = - v_{17} / v_{11}$.


Найдём, чему равен $RSS_1$ в этой регрессии.
\[
RSS_1 = e_1^T e_1 = x_{1*}^T x_{1*} / v_{11}^2  = v_{11} / v_{11}^2 = 1  / v_{11}.
\]

Бинго! Ура, $v_{11} = 1/ RSS_1$. 
Заметим, что $x_{1*} = e_1 v_{11} = e_1/RSS_1$.
Типичный недиагональный элемент равен $v_{12} = -\hb_{12} v_{11} = -\hb_{12}/RSS_1$.
И, наконец, матрицу-мать всех регрессий для $k=3$ в студию:
\[
V = (X^TX)^{-1} = \begin{pmatrix}
    1/RSS_1 & -{\hb_{12}}/{RSS_1} & -{\hb_{13}}/{RSS_1}\\
    - {\hb_{21}}/{RSS_2} & 1/RSS_2 & - {\hb_{23}}/{RSS_2} \\
    - {\hb_{31}}/{RSS_3}  & - {\hb_{32}}/{RSS_3} & 1/RSS_3 \\
\end{pmatrix} 
\]
Например, первая строка этой матрицы несёт всю информацию о регрессии $x_1$ на $x_2$ и $x_3$.

В случае пары регрессоров вспомним, что $R^2$ одинаков в регрессии $x_1$ на $x_2$ и в обратной и равен $R^2 = \hb_{12} \hb_{21}$.
Поэтому, $v_{12}^2 = v_{21}^2 = \hb_{12}\hb_{21} / (RSS_1 RSS_2) = R^2 / (RSS_1 RSS_2)$ и $V = (X^TX)^{-1}$ равна
\[
\begin{pmatrix}
    1/RSS_1 & -{\hb_{12}}/{RSS_1} \\
    - {\hb_{21}}/{RSS_2} & 1/RSS_2 \\
\end{pmatrix}  = 
\begin{pmatrix}
    \frac{1}{RSS_1} & -\frac{\abs{R}}{\sqrt{RSS_1 RSS_2}} \\
    -\frac{\abs{R}}{\sqrt{RSS_1 RSS_2}} & \frac{1}{RSS_2} \\
\end{pmatrix} = 
\frac{1}{1 - R^2}
\begin{pmatrix}
    \frac{1}{TSS_1} & -\frac{\abs{R}}{\sqrt{TSS_1 TSS_2}} \\
    -\frac{\abs{R}}{TSS_1 TSS_2} & \frac{1}{TSS_2} \\
\end{pmatrix}. 
\]

\end{document}

